\exercise{Model Learning}
The new and improved Spinbot 2000 is a multi-purpose robot platform.
It is made of a kinematic chain consisting of a linear axis $q_{1}$, a rotational axis $q_{2}$ and another linear axis $q_{3}$, as shown in the figure below.
These three joints are actuated with forces and torques of $u_{1}$, $u_{2}$, and $u_{3}$.
Different end effectors, including a gripper or a table tennis racket, can be mounted on the end of the robot, indicated by the letter $E$.
Thanks to Spinbot's patented SuperLight technology, the robot's mass is distributed according to one point mass of $m_{1}$ at the second joint and another point mass of $m_{2}$ at the end of the robot $E$.

\includegraphics[width=0.5\textwidth]{img/spinbot.png}

The inverse dynamics model of the Spinbot is given as
\begin{align*}
u_{1} &= (m_{1}+m_{2})(\ddot{q}_{1}+g),\\
u_{2} &= m_{2}(2\dot{q}_{3}\dot{q}_{2}q_{3}+q_{3}^{2}\ddot{q}_{2}),\\
u_{3} &= m_{2}(\ddot{q}_{3}-q_{3}\dot{q}_{2}^{2}).
\end{align*}
We now collected 100 samples from the robot while using a PD controller with gravity compensation at a rate of 500Hz.
The collected data (see \texttt{spinbotdata.txt}) is organized as follows\\
\begin{tabular}{| l || c | c | c | l  }
  \hline
   & $t_1$ & $t_2$ & $t_3$ & \ldots\\
  \hline
  \hline
  $q_{1}[m]$ &  &  &  & \\
  \hline
  $q_{2}[rad]$ &  &  &  & \\
  \hline
  $q_{3}[m]$ &  &  &  & \\
  \hline
  \ldots &  &  &  &  \\
  \hline
  $\ddot{q}_{3}[m/s^{2}]$ &  &  &  & \\
  \hline
  $u_{1}[N]$ &  &  &  & \\
  \hline
  $u_{2}[Nm]$ &  &  &  & \\
  \hline
  $u_{3}[N]$ &  &  &  & \\
  \hline
\end{tabular}

Given this data, you want to learn the inverse dynamics of the robot to use a model-based controller.
The inverse dynamics of the system will be modeled as $\vec{u}=\vec{\phi}(\vec{q},\dot{\vec{q}},\ddot{\vec{q}})^{\T}\vec{\theta}$, where $\vec{\phi}(\vec{q},\dot{\vec{q}},\ddot{\vec{q}})$ are features and $\vec{\theta}$ are the parameters.

\begin{questions}

%----------------------------------------------

\begin{question}{Problem Statement}{2}
What kind of machine learning problem is learning an inverse dynamics model? What kind of information do you need to solve such a problem?

\begin{answer}
	Theory:\\
	Learning Inverse Dynamics is quite interesting, as rigid body("Starrk\"orper") dynamics are lacking of good friction models and are therefore incomplete. Moreover dynamic parameters are difficult to estimate. The task of an inverse model is to predict the action needed to reach a desired state. This means a Inverse Matrix can be learned directly.
	
	Inverse Dynamics: $u=f(q,\dot{q},\ddot{q}_{ref}), \quad \dddot{q}_t^{des}=K_P(q_t^{des}-q_t)+K_D(\dot{q}_t^{des}-\dot{q}_t)$ (PD-Controller)
	
	Task:\\
	Kind of machine learning problem:\\
	Linear Regression? 
	
	Kind of information:\\
	In order to learn an inverse dynamics model you need a lot of measurements. In this particular case:
	$\Phi$ and $Y$


	\end{answer}

\end{question}

%----------------------------------------------


\begin{question}{Assumptions}{5}
Which standard assumption has been violated by taking the data from trajectories?

\begin{answer}
	Discrete time?
	\end{answer}

\end{question}

%----------------------------------------------


\begin{question}{Features and Parameters}{4}
Assuming that the gravity $g$ is unknown, what are the feature matrix $\vec{\phi}$ and the corresponding parameter vector $\vec{\theta}$ for $\vec{u}=[u_{1},u_{2},u_{3}]^{\T}$?
(Hint: you do not need to use the data at this point)

\begin{answer}
	
	What are the parameters?
	
	Is g=0 ? should I find the parameters m1,m2? by a lot of measurements?
	
		\begin{align*}
		u_{1} &= (m_{1}+m_{2})(\ddot{q}_{1}+g),\\
		u_{2} &= m_{2}(2\dot{q}_{3}\dot{q}_{2}q_{3}+q_{3}^{2}\ddot{q}_{2}),\\
		u_{3} &= m_{2}(\ddot{q}_{3}-q_{3}\dot{q}_{2}^{2}).
		\end{align*}
		
	$\underset{3\times 1} y= \underset{3\times 2} \phi^T \underset{2\times 1} \theta= \begin{bmatrix}
	\ddot{q_1}&0\\0&2\dot{q_3}\dot{q_2}\dot{q_3}+q_3^2\ddot{q_2}\\0&\ddot{q_3}-q_3 \dot{q_2}^2
	\end{bmatrix}
	\begin{bmatrix}
	m_1+m_2\\m_2
	\end{bmatrix}$
	
	For n measurements: \\
	$\underset{3 \times n}Y=(y_1,...y_n)$
	
	$\underset{3 \times 2n}\Phi=(\phi_1,...\phi_n)$
	
	Linear Regression:\\
	$\theta = (\Phi^T \Phi)^{-1} \Phi^T Y$

	\end{answer}

\end{question}

%----------------------------------------------


\begin{question}{Learning the Parameters}{2}
You want to compute the parameters $\vec{\theta}$ minimizing the squared error between the estimated forces/torques and the actual forces/torques.
Write down the matrix equation that you would use to compute the parameters. For each matrix in the equation, write down its size.\\
Then, compute the least-squares estimate of the parameters $\vec{\theta}$ from the data and report the learned values.

\begin{answer}
	
	Task:\\
	$min_{\theta} \sum_{i=1}^n (u_{meas,i}-u_{pred,i})^2 =min_{\theta} \sum_{i=1}^n (u_{meas,i}-\phi^T_i \theta)^2 $
	
	
	\end{answer}

\end{question}

%----------------------------------------------


\begin{question}{Recovering Model Information}{4}
	Can you recover the mass properties $m_{1}$ and $m_{2}$ from your learned parameters? Has the robot learned a plausible inverse dynamics model? Explain your answers.
	
\begin{answer}\end{answer}
\end{question}

%----------------------------------------------


\begin{question}{Model Evaluation}{7}
Plot the forces and torques predicted by your model over time, as well as those recorded in the data and comment the results. Is the model accuracy acceptable? If not, how would improve your model? Use one figure per joint.

\begin{answer}\end{answer}

\end{question}

%----------------------------------------------


\begin{question}[bonus]{Models for Control}{4}
Name and describe three different learned models types and their possible applications.

\begin{answer}\end{answer}

\end{question}

%----------------------------------------------

\end{questions}
