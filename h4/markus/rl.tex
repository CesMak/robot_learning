\exercise{Reinforcement Learning}

\begin{questions}
	
%----------------------------------------------

\begin{question}{RL Exploration Strategies}{10}
	In which spaces can you perform exploration in RL? Discuss the two  exploration strategies applicable to RL.
	
\begin{answer}
Exploration can be performed in action Space and Parameter space.

Parameter Space:\\
In Episode Based Policy Search a distribution is learned over the parameters of the low level control policy. Exploration in parameter space allows more sophisticated strategies and is often very efficient for a small amount of parameters. 

A problem is the high variance of the returns, because one add different random variables. If one has a too high variance, the learned policy can become instable. Too avoid this, you can use the baseline trick for example.

An advantage is, that you do not have too know the structure of the optimization problem. So it is an "Black-Box Optimizer".

Action Space:\\
In the Episode based Policy Search you learn an full movement as episode and then get the reward for it. In Step-based Policy Search one decomposite the episode in single timesteps. So we get the reward for single state-actions pairs. This reward has a lower variance.

In Step Based Policy Search there is less variance in quality assessment. Moreover as the exploration happens in action space it is less likely to create unstable policies.
\end{answer}
\end{question}


%----------------------------------------------


\end{questions}